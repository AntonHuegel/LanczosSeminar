\begin{frame}
    \frametitle{Phasenfluid und kanonische Transformation}
    Zu sehen ist also
    \begin{itemize}
        \item Das Phasenfluid wird durch die Hamilton-Gleichungen beschrieben
        \item Zwei Zustände des Phasenfluid können durch kanonische Transformation in Bezug gebracht werden
        \item Für infinitessimal kleine Zustandsübergänge ist die Transformation explizit gegeben
        \item \emph{Die Bewegung des Phasenfluids ist eine stetige Abfolge kanonischer Transformationen}
    \end{itemize}
    
    \begin{displaymath}
        S(q_i,\ldots,q_n;Q_1,\ldots,Q_n;t), \quad t>0
    \end{displaymath}
    
\end{frame}

\begin{frame}
    Bisher:
    \begin{itemize}
        \item Wahl beliebiger generierender Funktion $S$
        \item Damit erhielten wir $B = -H$
    \end{itemize}
    \vspace{1cm}
    Nun Rückwärts:
    \begin{itemize}
        \item Hamilton-Funktion $H$ ist zu physikalischem Problem gegeben
        \item \begin{displaymath} -H = B = \frac{\partial S}{\partial t} \end{displaymath}
        \item Finde $S$
    \end{itemize}
    
\end{frame}


\begin{frame}
    Zu lösen ist also die partielle Differenzialgleichung

    \begin{displaymath}
        \frac{\partial S}{\partial t} + H = 0
    \end{displaymath} \\
        \vspace{1cm}
    $\longrightarrow$ Haben wir damit nicht wieder dasselbe Problem wie zu Beginn? 
\end{frame}

\begin{frame}
    Nicht ganz: \\
            \vspace{1cm}
    Hamilton findet eine Funktion, die diese Gleichung löst und damit ermöglicht, die kanonische Transformation explizit anzugeben:    
    \begin{center} \emph{Die charakteristische Funktion} \end{center}
\end{frame}

%\begin{frame}
%    \begin{align*}
%        \text{Prinzip d. kl. Wirkung: } \quad A &= 2 \int_{\tau_1}^{\tau_2} T~dt \\
%         &=  \int_{\tau_1}^{\tau_2} \sum_{i=1}^{n} p_i \dot{q}_i ~dt \\
%         &= \int_{\tau_1}^{\tau_2} \sum_{i=1}^{n} p_i ~ dq_i \\[5mm]
%        \text{Energieerhaltung: } \quad  H&(q_1,\ldots,q_n,p_1,\ldots,p_n) - E = 0 \\[5mm]
%      \Leftrightarrow  \text{Jacobi: } \quad \min A ~ \text{ mit } ~ A &= \sqrt{2} \int_{\tau_1}^{\tau_2} \sqrt{E-V} ~ d\bar{s}
%    \end{align*}
%\end{frame}

\begin{frame}
	\frametitle{Hamiltons charakteristische Funktion}
	
	\begin{itemize}
		\item Die Lösung von 
		\begin{displaymath}
			\frac{\partial S}{\partial t} + H = 0
		\end{displaymath}
		ist die kürzeste Verbindung auf der Energie-Ebene zwischen dem Anfangspunkt $q_1,\ldots,q_n$ und einem beliebigen weiteren Punkt $\bar{q}_1,\ldots,\bar{q}_n$ auf der zu $H$ gehörigen Trajektorie im Phasenraum.
		
		\item Eine solche Verbindungslinie wird auch Geodäte genannt
		
		\item Die Länge der gesuchten Geodäte bezeichnen wir mit $W=W(q_1,\ldots,q_n;\bar{q}_1,\ldots,\bar{q}_n;t)$;\\ $\rightarrow$ wird auch Hamiltons Grundfunktion (principal function) genannt.
	\end{itemize}	
	
\end{frame}


\begin{frame}
  \vspace*{-0.75cm} \hspace*{-2.3cm} \includegraphics[scale=0.4]{images/geodaete}
\end{frame}

\begin{frame}
    \begin{itemize}
        \item Hamiltons Grundfunktion liefert die Länge der Geodäte zwischen Anfangs- und Endpunkt der Trajektorie auf der Energie-Fläche im Phasenraum
        \item Diese Geodäte wird auch Hamiltons charakteristische Funktion genannt und ist die generierende Funktion für die zugehörige kanonische Transformation
        \item Die mit Hamiltons charakteristischer Funktion generierte kanonische Transformation beschreibt die Bewegung eines Partikels im Phasenfluid
     \end{itemize}
\end{frame}