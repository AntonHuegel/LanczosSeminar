\begin{frame}
    \frametitle{Phasenfluid und kanonische Transformation}
    Zu sehen ist also
    \begin{itemize}
        \item Das Phasenfluid wird durch die Hamilton-Gleichungen beschrieben
        \item Zwei Zustände des Phasenfluid können durch kanonische Transformation in Bezug gebracht werden
        \item Für infinitessimal kleine Zustandsübergänge ist die Transformation explizit gegeben
        \item \emph{Die Bewegung des Phasenfluids ist eine stetige Abfolge kanonischer Transformationen}
    \end{itemize}
    
\end{frame}

\begin{frame}
    Bisher:
    \begin{itemize}
        \item Wahl beliebiger generierender Funktion $S$
        \item Damit erhielten wir $B = -H$
    \end{itemize}
    \vspace{1cm}
    Nun Rückwärts:
    \begin{itemize}
        \item Hamilton-Funktion $H$ ist zu physikalischem Problem gegeben
        \item \begin{displaymath} -H = B = \frac{\partial S}{\partial t} \end{displaymath}
        \item Finde $S$
    \end{itemize}
    
\end{frame}