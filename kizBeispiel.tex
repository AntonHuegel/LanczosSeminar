\documentclass{beamer}
\mode<presentation>
{
  \usetheme{myulm}
  \setbeamercovered{transparent}
  \setbeamertemplate{navigation symbols}{} % no navigation bar
  \setbeamersize{sidebar width left=1.17cm}
}

\usepackage[ngerman]{babel}
\usepackage[utf8]{inputenc}
\usepackage{amsmath,amssymb,amsfonts}
\usepackage{times}
\usepackage{graphicx}
\usepackage{fancyvrb}
\usepackage{array}
\usepackage{colortbl}

% Anfang der Titelfolie
% Anpassung von: Titel, Untertitel, Autor, Datum und Institut

\title{Hier steht der Titel der Pr\"{a}sentation, auch zweizeilig m\"{o}glich}
\subtitle{Manchmal wird ein Untertitel ben\"{o}tigt, der ebenfalls einzeilig oder mehrzeilig sein kann}
\author{Vorname Name}
\newcommand{\presdatum}{\today} % alternativ zu \today: Eingabe eines festen Datums
\institute
{Institut, Einrichtung\\}
%Ende der Titelfolie

% Anfang der Kopfzeile der Folien
% Anpassung von: Zwischentitel, Leitthema oder Name
% Das Datum wird oben geändert: unter \presdatum{}!

\newcommand{\zwischentitel}{Zwischentitel}
\newcommand{\leitthema}{Leitthema oder Name}
% Ende der Kopfzeile

% Anfang der Folien
\begin{document}
\hspace*{-1.49cm}
\frame[plain]{\titlepage}

% Das Inhaltsverzeichnis
\hspace*{-0.7cm}
\begin{frame}
  \frametitle{Inhaltsverzeichnis}
  \tableofcontents
\end{frame}


% 1. Folie
\section{Vortragsabschnitt 1}
\begin{frame}
  \frametitle{Entropie}
\vspace{-2.6cm}
  \begin{itemize}
    \item Das ist eine M\"{o}glichkeit f\"{u}r eine Aufz\"{a}hlung
    \item Aufz\"{a}hlung 2
    \item Aufz\"{a}hlung 3
  \end{itemize}
\end{frame}

% 2. Folie
\begin{frame}
  \frametitle{Entropie - Definition}
\vspace{-2.6cm}
  \begin{itemize}
    \item Das ist eine M\"{o}glichkeit f\"{u}r eine Aufz\"{a}hlung
    \item Aufz\"{a}hlung 2
    \item Aufz\"{a}hlung 3
  \end{itemize}
\end{frame}

% 3. Folie
\section{Vortragsabschnitt 2}
\begin{frame}
  \frametitle{\"{U}berschrift 1}
\vspace{-2.6cm}
  \begin{itemize}
    \item Das ist eine M\"{o}glichkeit f\"{u}r eine Aufz\"{a}hlung
    \item Aufz\"{a}hlung 2
    \item Aufz\"{a}hlung 3
  \end{itemize}
\end{frame}
   
% 4. Folie
\section{Vortragsabschnitt n}
\begin{frame}
  \frametitle{\"{U}berschrift 2}
 \vspace{-3.5cm}
 \hspace{-0.8cm}
  Das ist ein Beispieltext in Arial 14 pt
\end{frame}

% 5. Folie
\section{Die Uni-Farben im \"{U}berblick}
\begin{frame}
 \frametitle{Die Uni-Farben im \"{U}berblick}
 \vspace{-0.1cm}
\hspace{-0.75cm}  \tiny{Mathematik/Wirtschaftswissenschaften sRGB 100\%: 86-170-28, Prozentwert von 100\% - 10\% in 10er-Schritten} \\
\hspace{-0.7cm}  \colorbox{farbwert-mawi}{}{} \\
 \vspace{0.5cm}
\hspace{-0.7cm}  \tiny{Ingenieurwissenschaften/Informatik sRGB 100\%: 163-38-56, Prozentwerte von 100\% - 10\% in 10er-Schritten} \\
\hspace{-0.7cm}  \colorbox{farbwert-inwiin}{}{} \\
  \vspace{0.5cm}
\hspace{-0.7cm}  \tiny{Naturwissenschaften sRGB 100\%: 163-38-56, Prozentwerte von 100\% - 10\% in 10er-Schritten} \\
\hspace{-0.7cm}  \colorbox{farbwert-nawi}{}{} \\
   \vspace{0.5cm}
   \hspace{-0.7cm}  \tiny{Medizin sRGB 100\%: 38-84-124, Prozentwerte von 100\% - 10\% in 	       10er-Schritten} \\
\hspace{-0.7cm}  \colorbox{farbwert-med-blau}{}{}  \\
  \vspace{0.5cm}
\hspace{-0.7cm}  \tiny{Farbwerte Uni-Blau sRGB 100\%: 125-154-170, Prozentwerte von 100\% - 10\% in 	       10er-Schritten} \\
\hspace{-0.7cm}  \colorbox{farbwert-blau}{}{}  \\
  \vspace{0.5cm}
  \hspace{-0.7cm}  \tiny{Farbwert Beige sRGB 100\%: 169-162-141, Prozentwerte von 100\% - 10\% in 10er-Schritten} \\
\hspace{-0.7cm}  \colorbox{farbwert-beige}{}{} \\
  \vspace{0.5cm}
\end{frame}
%Ende der Folien

\end{document}