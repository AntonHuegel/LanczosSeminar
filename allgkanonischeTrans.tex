\begin{frame}
    Die Invarianz von    
    \begin{displaymath}
            \sum_{i=1}^n p_i \delta q_i = \sum_{i=1}^n P_i \delta Q_i
    \end{displaymath}
    ist nicht notwendig. \\
    
    Allgemeinerer Ansatz:
    \begin{displaymath}
    \sum_{i=1}^n p_i \delta q_i = \sum_{i=1}^n P_i \delta Q_i + \delta S
    \end{displaymath}

\end{frame}

\begin{frame}

    Das kanonische Integral sieht nun folgendermaßen aus:
    
    \begin{align*}
    A &= \int_{t_1}^{t_2} \left( \sum_{i=1}^n p_i dq_i - H dt \right) \\
      &= \int_{t_1}^{t_2} \left( \sum_{i=1}^n P_i dQ_i - H dt \right)  +  \underbrace{\int_{t_1}^{t_2} dS}_{=\text{const.}}
    \end{align*}
    
    $\Longrightarrow$  Kannonische Gleichungen invariant unter dieser Transformation.
    
\end{frame}

\begin{frame}
    \frametitle{Allgemeine kanonische Transformation}
    
    
    \begin{displaymath}
        \sum_{i=1}^{n} (p_i \delta q_i - P_i \delta Q_i) = \delta S
    \end{displaymath}
    
    \begin{center} mit der generierenden Funktion $S = S(q_1,\ldots,q_n;Q_1,\ldots,Q_n)$ \end{center}
    
\end{frame}