\begin{frame}
    \frametitle{Lagrange- und Poisson-Klammer}

    Seien $q_i$ und $p_i$ zwei Koordinatensysteme, $u$ und $v$ Parameter. \\

    \begin{align*}
    \textbf{Lagrange-Klammer: } \qquad    [u,v] = \sum_{i=1}^{n} \left(   \frac{\partial q_i}{\partial u} \frac{\partial p_i}{\partial v} - \frac{\partial q_i}{\partial v} \frac{\partial p_i}{\partial u}  \right)  \\
    \textbf{Poisson-Klammer: } \qquad    (u,v) = \sum_{i=1}^{n} \left(   \frac{\partial u}{\partial q_i} \frac{\partial v}{\partial p_i} - \frac{\partial v}{\partial q_i} \frac{\partial u}{\partial p_i}  \right)  
    \end{align*}

\end{frame}

\begin{frame}
    Es gilt:
    \vspace{5mm}
    \begin{itemize}
        \item Die kanonischen Transformationen sind diejenigen Transformationen von $q_i,p_i$ zu $Q_i,P_i$, für welche die \emph{Lagrange-Klammer} invariant ist.
        \vspace{5mm}
        \item Die kanonischen Transformationen sind diejenigen Transformationen von $q_i,p_i$ zu $Q_i,P_i$, für welche die \emph{Poisson-Klammer} invariant ist.
    \end{itemize}
    
\end{frame}
