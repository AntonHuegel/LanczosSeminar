\documentclass{beamer}
\mode<presentation>
{
  \usetheme{myulm}
  \setbeamercovered{transparent}
  \setbeamertemplate{navigation symbols}{} % no navigation bar
  \setbeamersize{sidebar width left=1.17cm}
}

\usepackage[ngerman]{babel}
\usepackage[utf8]{inputenc}
\usepackage{amsmath,amssymb,amsfonts}
\usepackage{times}
\usepackage{graphicx}
\usepackage{fancyvrb}
\usepackage{array}
\usepackage{colortbl}

% Anfang der Titelfolie
% Anpassung von: Titel, Untertitel, Autor, Datum und Institut

\title{Kanonische Transformationen und die Hamilton-Jacobi-Gleichung}
\subtitle{Seminar I für Computationals Science and Engineering bei Prof. Lebiedz}
\author{Alexander Dürr und Anton Hügel}
\newcommand{\presdatum}{8. Februar 2018} % alternativ zu \today: Eingabe eines festen Datums
\institute
{\\Universität Ulm, Institut für Numerische Mathematik}
%Ende der Titelfolie

% Anfang der Kopfzeile der Folien
% Anpassung von: Zwischentitel, Leitthema oder Name
% Das Datum wird oben geändert: unter \presdatum{}!

\newcommand{\zwischentitel}{Zwischentitel}
\newcommand{\leitthema}{Lanczos Kap. 7 und 8}
% Ende der Kopfzeile

% Anfang der Folien
\begin{document}
\hspace*{-1.49cm}
\frame[plain]{\titlepage}

% Das Inhaltsverzeichnis
\hspace*{-0.7cm}
\begin{frame}
  \frametitle{Inhaltsverzeichnis}
  \tableofcontents
\end{frame}


\section{Bekanntes: Die kanonischen Gleichungen}

\section{Kanonische Transformationen}
    \begin{frame}
    \frametitle{Gesucht: Lösung der kanonischen Gleichungen}
    
    Gesucht werden die Lösungen $p_k$, $q_k$ zu
    \begin{align*}
        \dot{q}_k &= \frac{\partial H}{\partial p_k} \\
        \dot{p}_k &= -\frac{\partial H}{\partial q_k}.
    \end{align*}
    
    Jedoch: In dieser Form nicht lösbar.
    
\end{frame}

    \subsection{Lösung mechanischer Probleme mittels Koordinatentranformation}
    \begin{frame}
    \frametitle{Ausweg: Koordinatentransformation}
    
    Gesucht wird ein Koordinatensystem, indem die kanonischen Gleichungen direkt gelöst werden können.
    
    \begin{figure}
        \includegraphics[scale=0.2]{images/koordtrans.png}
    \end{figure}
    
    
\end{frame}

\begin{frame}
    \frametitle{Ausweg: Koordinatentransformation}
    
    \textbf{Wichtig:} Lösungen müssen erhalten bleiben.\\
    D.h. wir brauchen Transformationen, gegenüber derer die kanonischen Gleichungen invariant sind. \\
    \vspace{5mm}    
    Solche Transformationen werden \emph{kanonische Transformationen} genannt.
    
    \begin{figure}
        \includegraphics[scale=0.2]{images/koordtrans.png}
    \end{figure}
    
    
\end{frame}
    
    \subsection{Die Langrange'sche Punkttransformation}
    \begin{frame}
    \frametitle{Punkttransformation}
    
    Alte Koordinaten: $p_k, q_k$ \\
    Neue Koordinaten: $P_k, Q_k$
    
    Punkttransformationen haben die Form
    
    \begin{align*}
    q_1 &= f_1(Q_1,\ldots,Q_n) \\
        &\quad\vdots \\
    q_n &= f_n(Q_1,\ldots,Q_n)    
    \end{align*}
    
\end{frame}

\begin{frame}
    \frametitle{Punkttransformation}
    
    Alte Koordinaten: $p_k, q_k$ \\
    Neue Koordinaten: $P_k, Q_k$
    
    Punkttransformationen haben die Form
    
    \begin{align*}
    q_1 &= f_1(Q_1,\ldots,Q_n) \\
    &\quad\vdots \\
    q_n &= f_n(Q_1,\ldots,Q_n)    
    \end{align*}
    
\end{frame}

\begin{frame}
    \frametitle{Kanonischer Integrand}
    
    Für eine Transformation gilt:
        \begin{align*}
        \dot{q}_k &= \frac{\partial H}{\partial p_k} \\
        \dot{p}_k &= -\frac{\partial H}{\partial q_k}		
        \end{align*}
      \begin{center}  invariant \end{center} 
        
        \begin{displaymath}
        \Longleftrightarrow
        \end{displaymath}
        
      \begin{center} Lagrange-Funktion \end{center} 
        \begin{displaymath}
        L = \sum_{i=1}^n p_i \dot{q}_i - H
        \end{displaymath}
      \begin{center}  ist invariant. \end{center} 

\end{frame}

\begin{frame}
    \begin{displaymath}
    \sum_{i=1}^n p_i \dot{q}_i = H \qquad \text{invariant}
    \end{displaymath}
    
    \begin{displaymath}
    \Longleftarrow \qquad \sum_{i=1}^n p_i \partial q_i = \sum_{i=1}^n P_i \partial Q_i
    \end{displaymath}
    
    für infinitesimale Änderungen der $q_i$.
    
\end{frame}

\begin{frame}
    Das kanonische Integral 
    
    \begin{displaymath}
    A = \int_{t_1}^{t_2} \left( \sum_{i=1}^n p_i dq_i - H dt \right) 
    \end{displaymath}
    
    kann in diesem Fall durch
    
    \begin{displaymath}
    A = \int_{t_1}^{t_2} \left( \sum_{i=1}^n P_i dQ_i - H dt \right) 
    \end{displaymath}
    
    ersetzt werden. \\
    \vspace{5mm}
    Die Hamilton-Funktion $H$ ist also Invariante dieser Transformation.
    
\end{frame}


\begin{frame}
   Die Transformation ist dann implizit gegeben:\\

    \begin{displaymath}
        \partial q_i = \frac{\partial f_i}{\partial Q_1} \partial Q_1 + \ldots +  \frac{\partial f_i}{\partial Q_n} \partial Q_n \qquad i=1,\ldots,n
    \end{displaymath} \\
   \vspace{5mm}      
    \textbf{Wichtig:} Diese Einschränkungen sind hinreichend aber nicht notwendig für eine kanonische Transformation.
\end{frame}
    
    \subsection{Die allgemeine kanonische Transformation}
    \begin{frame}
    \frametitle{Die allgemeine kanonische Transformation}
    Die Invarianz von    
    \begin{displaymath}
            \sum_{i=1}^n p_i \partial q_i = \sum_{i=1}^n P_i \partial Q_i
    \end{displaymath}
    ist nicht notwendig. \\
    
    Allgemeinerer Ansatz:
    \begin{displaymath}
    \sum_{i=1}^n p_i \partial q_i = \sum_{i=1}^n P_i \partial Q_i + \partial S
    \end{displaymath}

\end{frame}

\begin{frame}

    Das kanonische Integral sieht nun folgendermaßen aus:
    
    \begin{align*}
    A &= \int_{t_1}^{t_2} \left( \sum_{i=1}^n p_i dq_i - H dt \right) \\
      &= \int_{t_1}^{t_2} \left( \sum_{i=1}^n P_i dQ_i - H dt \right)  +  \int_{t_1}^{t_2} dS
    \end{align*}
    
\end{frame}
    
    \subsection{Die bilineare Differentialform}
    \begin{frame}
    \frametitle{Die blineare Differenzialform}
    \begin{itemize}
        \item Jeder Transformation lässt bestimmte Größen unverändert
        \item Diese Invarianten bestimmen Eigenschaften der Transformation
        \item Für kanonische Transformation hatten wir zunächst $\sum p_i \delta q_i$
        \item Dann fanden wir die allgemeinere Bedingung     
                \begin{displaymath}
                \sum_{i=1}^{n} (p_i \delta q_i - P_i \delta Q_i) = \delta S
                \end{displaymath}
              Welcher Invarianten entspricht diese Bedingung?
    \end{itemize}
\end{frame}

\begin{frame}
   
        \begin{displaymath}
        \sum_{i=1}^{n} p_i \delta q_i - \sum_{i=1}^{n} P_i \delta Q_i = \delta S
        \end{displaymath}
        
        $\longrightarrow$ Erinnert an die Arbeit in einem monogenen (monogenic) System.\\
        \vspace{3mm}
        \emph{Masseteilchen wird auf beliebigem geschlossenen Pfad bewegt.\
              Ist die verrichtete Arbeit Null, wenn man wieder am Ausgangspunkt ankommt, so nennt man das System monogen.}
            
      \begin{center} \includegraphics[scale=0.15]{images/monogenicSys}  \end{center}  
        
        

\end{frame}

\begin{frame}
    Integration von
    \begin{displaymath}
    \sum_{i=1}^{n} p_i d q_i - \sum_{i=1}^{n} P_i d Q_i = d S
    \end{displaymath}
    entlang einer geschlossenen Kurve ergibt:
   \begin{displaymath}
   \oint \sum_{i=1}^{n} p_i d q_i - \oint \sum_{i=1}^{n} P_i d Q_i = 0
   \end{displaymath}
    \begin{center} \includegraphics[scale=0.225]{images/circulation}  \end{center}  

\end{frame}

\begin{frame}
     \emph{Für jede geschlossene Kurve im Phasenraum ist}
    \begin{displaymath}
    \Gamma = \oint \sum_{i=1}^{n} p_i d q_i = \oint \sum_{i=1}^{n} P_i d Q_i
    \end{displaymath}
     \emph{eine Invariante gegenüber der kanonischen Transformationen.}
 
\end{frame}
    
    \subsection{Lagrange Klammer}
    \begin{frame}
    \frametitle{Lagrange- und Poisson-Klammer}

    Seien $q_i$ und $p_i$ zwei Koordinatensysteme, $u$ und $v$ Parameter. \\

    \begin{align*}
    \textbf{Lagrange-Klammer: } \qquad    [u,v] = \sum_{i=1}^{n} \left(   \frac{\partial q_i}{\partial u} \frac{\partial p_i}{\partial v} - \frac{\partial q_i}{\partial v} \frac{\partial p_i}{\partial u}  \right)  \\
    \textbf{Poisson-Klammer: } \qquad    (u,v) = \sum_{i=1}^{n} \left(   \frac{\partial u}{\partial q_i} \frac{\partial v}{\partial p_i} - \frac{\partial v}{\partial q_i} \frac{\partial u}{\partial p_i}  \right)  
    \end{align*}

\end{frame}

\begin{frame}
    Es gilt:
    \vspace{5mm}
    \begin{itemize}
        \item Die kanonischen Transformationen sind diejenigen Transformationen von $q_i,p_i$ zu $Q_i,P_i$, für welche die \emph{Lagrange-Klammer} invariant ist.
        \vspace{5mm}
        \item Die kanonischen Transformationen sind diejenigen Transformationen von $q_i,p_i$ zu $Q_i,P_i$, für welche die \emph{Poisson-Klammer} invariant ist.
    \end{itemize}
    
\end{frame}

    
    \subsection{Gruppeneigenschaft}
    \input{Gruppeneigenschaft}
    
    \subsection{Infinitesimale kanonische Transformationen}
    
    \subsection{Das Phasenfluid als kanonische Transformationen}
    
    
\section{Die Hamilton-Jacobi-Gleichung}

    \subsection{Jacobis Transformationstheorie}
    
    \subsection{Lösung durch Separation}
    
    \subsection{Partielle Differenzialgleichungen bei Hamilton und Jacobi}

    \subsection{Geometrische Lösung und Wellenanalogie}

\section{Zusammenfassung}


\end{document}