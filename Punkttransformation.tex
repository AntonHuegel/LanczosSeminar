\begin{frame}
    \frametitle{Punkttransformation}
    
    Alte Koordinaten: $p_k, q_k$ \\
    Neue Koordinaten: $P_k, Q_k$
    
    Punkttransformationen haben die Form
    
    \begin{align*}
    q_1 &= f_1(Q_1,\ldots,Q_n) \\
        &\quad\vdots \\
    q_n &= f_n(Q_1,\ldots,Q_n)    
    \end{align*}
    
\end{frame}

\begin{frame}
    \frametitle{Punkttransformation}
    
    Alte Koordinaten: $p_k, q_k$ \\
    Neue Koordinaten: $P_k, Q_k$
    
    Punkttransformationen haben die Form
    
    \begin{align*}
    q_1 &= f_1(Q_1,\ldots,Q_n) \\
    &\quad\vdots \\
    q_n &= f_n(Q_1,\ldots,Q_n)    
    \end{align*}
    
\end{frame}

\begin{frame}
    \frametitle{Kanonischer Integrand}
    
    Für eine Transformation gilt:
        \begin{align*}
        \dot{q}_k &= \frac{\partial H}{\partial p_k} \\
        \dot{p}_k &= -\frac{\partial H}{\partial q_k}		
        \end{align*}
      \begin{center}  invariant \end{center} 
        
        \begin{displaymath}
        \Longleftrightarrow
        \end{displaymath}
        
      \begin{center} Lagrange-Funktion \end{center} 
        \begin{displaymath}
        L = \sum_{i=1}^n p_i \dot{q}_i - H
        \end{displaymath}
      \begin{center}  ist invariant. \end{center} 

\end{frame}

\begin{frame}
    \begin{displaymath}
    \sum_{i=1}^n p_i \dot{q}_i = H \qquad \text{invariant}
    \end{displaymath}
    
    \begin{displaymath}
    \Longleftarrow \qquad \sum_{i=1}^n p_i \partial q_i = \sum_{i=1}^n P_i \partial Q_i
    \end{displaymath}
    
    für infinitesimale Änderungen der $q_i$.
    
\end{frame}

\begin{frame}
    Das kanonische Integral 
    
    \begin{displaymath}
    A = \int_{t_1}^{t_2} \left( \sum_{i=1}^n p_i dq_i - H dt \right) 
    \end{displaymath}
    
    kann in diesem Fall durch
    
    \begin{displaymath}
    A = \int_{t_1}^{t_2} \left( \sum_{i=1}^n P_i dQ_i - H dt \right) 
    \end{displaymath}
    
    ersetzt werden. \\
    \vspace{5mm}
    Die Hamilton-Funktion $H$ ist also Invariante dieser Transformation.
    
\end{frame}


\begin{frame}
   Die Transformation ist dann implizit gegeben:\\

    \begin{displaymath}
        \partial q_i = \frac{\partial f_i}{\partial Q_1} \partial Q_1 + \ldots +  \frac{\partial f_i}{\partial Q_n} \partial Q_n \qquad i=1,\ldots,n
    \end{displaymath} \\
   \vspace{5mm}      
    \textbf{Wichtig:} Diese Einschränkungen sind hinreichend aber nicht notwendig für eine kanonische Transformation.
\end{frame}